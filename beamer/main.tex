%%%%%%%%%%%%%%%%%%%%%%%%%%%%%%%%%%%%%%%%%%%%%%%%%%%%%%%%%%%%
%%  This Beamer template was created by Cameron Bracken.
%%  Anyone can freely use or modify it for any purpose
%%  without attribution.
%%
%%  Last Modified: January 9, 2009
%%

\documentclass[xcolor=x11names,compress]{beamer}

%% General document %%%%%%%%%%%%%%%%%%%%%%%%%%%%%%%%%%
\usepackage{listings}
\usepackage{epstopdf}
\usepackage{graphicx}
\usepackage{tikz}
\usepackage[french]{babel}
\usepackage[T1]{fontenc}
\usepackage[utf8]{inputenc}
\usepackage{array}
\usepackage{svg}
\usepackage{calc}
\usetikzlibrary{decorations.fractals, calc}
%%%%%%%%%%%%%%%%%%%%%%%%%%%%%%%%%%%%%%%%%%%%%%%%%%%%%%

\definecolor{bg-color}{HTML}{F26E27}
\definecolor{green-avantages}{HTML}{23B854}
\newcommand{\firstlogo}{images/m2dl.png}
\newcommand{\secondlogo}{images/ups.jpg}
\newcommand{\footsubject}{Git, essayons de reprendre le contrôle !} % Subject in footer

%% Beamer Layout %%%%%%%%%%%%%%%%%%%%%%%%%%%%%%%%%%
\mode<presentation> {
	\usepackage{../../../LaTeX-Beamer-Theme/beamerthemeShortPresentation}
}
%%%%%%%%%%%%%%%%%%%%%%%%%%%%%%%%%%%%%%%%%%%%%%%%%%

\setbeamercovered{invisible}

\AtBeginSection[]{%
  \begin{frame}<beamer>
  \tableofcontents[currentsection]
  \end{frame}
}

%\input{listings.tex}

\begin{document}


%%%%%%%%%%%%%%%%%%%%%%%%%%%%%%%%%%%%%%%%%%%%%%%%%%%%%%
%%%%%%%%%%%%%%%%%%%%%%%%%%%%%%%%%%%%%%%%%%%%%%%%%%%%%%
\begin{frame}
\title{\centering Git, Essayons de reprendre le contrôle !}
\subtitle{}
\author{
	\centering
  Antoine de {\sc Roquemaurel}\\
	  \begin{tabular}{ccc}
		  \includegraphics[height=7pt]{images/twitter.png}~satenske
	  \end{tabular}
	  \\
  {\footnotesize
	  \vspace{10px}
		Développeur Java consultant chez Tech Advantage\\
  }
}
\date{
	{
	  \begin{tabular}{ccc}
		  \includegraphics[height=1.cm]{images/logo_extia}\hspace{20px} & 
		  \includegraphics[height=1.cm]{images/meetup_by_extia} & 
		  \hspace{10px} \includegraphics[height=1cm]{images/logo_git}
	  \end{tabular}
	}
	\\
	\vspace{10px}
	Meetup Java / C\# du 28 Mars 2019\\\vspace{0.9cm}
	\vfill
	\includegraphics[width=1cm]{images/cc-by.eps} 
	\begin{minipage}{0.8\textwidth} 
		\tiny Cette œuvre est mise à disposition selon les termes de la Licence Creative Commons By 4.0\\
	\end{minipage}
}
\titlepage
\end{frame}
\begin{frame}{Le logiciel de gestion de versions}
	\begin{figure}[H]
		\includegraphics[width=9.4cm]{images/intro/cvs.png}
		\vspace{-10px}
		\caption{Pourquoi devrais-je utiliser le contrôle de version ?\footnote{\tiny\url{https://stackoverflow.com/questions/1408450/why-should-i-use-version-control}}}
	\end{figure}
\end{frame}

\begin{frame}{Git}
	\vspace{-1cm}
	\begin{tabular}{ll}
		\begin{minipage}{4.0cm}
		\includegraphics[width=4cm]{images/linus.jpg} 
		\end{minipage}
		&
		\begin{minipage}{0.6\textwidth}
		\vspace{60px}
			\begin{itemize}
				\item<1-> Créé en 2005 par {Linus Torvalds}
				\item<1-> Décentralisé
				\item<1-> Excellente gestion des branches
				\item<1-> Efficace sur de gros projet
					\uncover<2> {
					\begin{itemize}
						\item Microsoft Windows : 
					\tiny{
					\begin{itemize}
						\item 3 500 000 fichiers, soit 300 Go
						\item 440 branches
						\item 4 000 utilisateurs 
						\item 10 000 merges 
					\end{itemize}
					}
					\end{itemize}
					}
			\end{itemize}
		\end{minipage}
	\end{tabular}

	\uncover<3>{
		\vfill
		\centering
		<< \textit{I'm an egotistical bastard, and I name all my projects after myself. First 'Linux', now 'git'.} >>
		\vspace{30px}
	}
\end{frame}

\section{Les bases}
\begin{frame}{Le système décentralisé}
	
\end{frame}

\begin{frame}{La zone de transit (\textit{staging area})}
	\begin{figure}[H]
		\vspace{-10px}
		\includegraphics[width=11cm]{images/section1/staging.png}
		\caption{Fonctionnement de Git}
	\end{figure}	
\end{frame}
\begin{frame}{Les actions de base}
\begin{itemize}
	\item \texttt{commit}
	\item \texttt{fetch}
	\item \texttt{push}
	\item \texttt{blame}
\end{itemize}
\end{frame}
% Git débite à la hash
\section{Le workflow Git}
% Restons branchés
\section{La collaboration}

\appendix
\section*{Conclusion}
\begin{frame}
\only<1,3>{
	\begin{figure}[H]
		\centering
		\vspace{-7px}
		\includegraphics[width=9cm]{images/git-mess.png}
		\vspace{-10px}
		\caption{\textit{So, you have a mess on your hands ?}\footnote{\url{http://justinhileman.info/article/git-pretty/}}}
	\end{figure}
	}

	\only<2>{
	{
		\vspace{10.08cm}
		\hspace{0.5cm}
        \begin{tikzpicture}[remember picture,overlay]
            \node[at=(current page.center)] {
                \includegraphics[keepaspectratio,
                                 height=1.05\paperheight]{images/git-mess.png}
            };
        \end{tikzpicture}
		}
		}
\end{frame}

\section*{Références}
\begin{frame}{Références}
		\centering
	\begin{tabular}{cc}
		\includegraphics[height=4cm]{images/refs/oreally.jpg}&
		\includegraphics[height=4cm]{images/refs/progit.png}\\
	\end{tabular}
	\vfill
	\begin{itemize}
		\item \footnotesize \texttt{\href{https://git-scm.com}{git-scm.com}}\\
			{Site officiel}
		\item \footnotesize \texttt{\href{https://learngitbranching.js.org}{learngitbranching.js.org}}\\
			{Apprendre Git de manière ludique}
		\item \footnotesize \texttt{\href{https://github.com/aroquemaurel/Presentation-beamer-Git}{github.com/aroquemaurel/Presentation-beamer-Git}}\\
			Les sources \LaTeX de cette présentation
	\end{itemize}
\end{frame}

\end{document}
